% (The MIT License)
%
% Copyright (c) 2023-2024 Yegor Bugayenko
%
% Permission is hereby granted, free of charge, to any person obtaining a copy
% of this software and associated documentation files (the 'Software'), to deal
% in the Software without restriction, including without limitation the rights
% to use, copy, modify, merge, publish, distribute, sublicense, and/or sell
% copies of the Software, and to permit persons to whom the Software is
% furnished to do so, subject to the following conditions:
%
% The above copyright notice and this permission notice shall be included in all
% copies or substantial portions of the Software.
%
% THE SOFTWARE IS PROVIDED 'AS IS', WITHOUT WARRANTY OF ANY KIND, EXPRESS OR
% IMPLIED, INCLUDING BUT NOT LIMITED TO THE WARRANTIES OF MERCHANTABILITY,
% FITNESS FOR A PARTICULAR PURPOSE AND NONINFRINGEMENT. IN NO EVENT SHALL THE
% AUTHORS OR COPYRIGHT HOLDERS BE LIABLE FOR ANY CLAIM, DAMAGES OR OTHER
% LIABILITY, WHETHER IN AN ACTION OF CONTRACT, TORT OR OTHERWISE, ARISING FROM,
% OUT OF OR IN CONNECTION WITH THE SOFTWARE OR THE USE OR OTHER DEALINGS IN THE
% SOFTWARE.

\documentclass[nobrand,anonymous,nodate,nosecurity]{huawei}
\usepackage{enumerate}
\usepackage{multicol}
\usepackage{wrapfig}
\usepackage{href-ul}
\usepackage{ffcode}
\usepackage{iexec}
\usepackage{paralist}
\newcommand\REG{$^{\tiny{\textsf{\textregistered}}}$}
\newcommand\TM{$^{\tiny{\textsf{TM}}}$}
\newcommand\angry[1]{\textcolor{red}{\emph{#1}}}
\begin{document}

\iexec[quiet]{wget -N --quiet https://www.yegor256.com/images/face-1024x1024.jpg}

{\sffamily{\bfseries\Large Open Source Best Practices (OSBP)}

Series of lectures by \href{https://www.yegor256.com}{Yegor Bugayenko} \\
To be presented to students of \href{https://innopolis.university/en/}{Innopolis University} in 2024
% and \href{https://www.youtube.com/playlist?list=PLaIsQH4uc08woJKRAA7mmjs9fU0jeKjjM}{video recorded}}
\\
The entire set of slide decks is in the \href{https://github.com/yegor256/osbp}{yegor256/osbp} GitHub repository

\begin{abstract}
In the course, students will learn how to interact with other programmers in open source GitHub repositories, ensuring that pull requests integrate seamlessly, reputation grows, the popularity of repositories increases, and the satisfaction of being an open source contributor materializes. This skill may also help students in their work with proprietary repositories, especially when teams are remotely distributed.
\end{abstract}

\textbf{What is the goal?}\\
This course introduces best practices for GitHub-based software development, which may eventually help students gain 10,000 stars in their own repositories.

\floatstyle{plain}
\restylefloat{figure}
\begin{wrapfigure}{r}{1.2in}%
\raggedleft%
\includegraphics[width=1.2in]{face-1024x1024.jpg}%
\end{wrapfigure}
\textbf{Who is the teacher?}\\
Yegor is developing software for more than 30 years, being a hands-on programmer
(see his GitHub account with 4.5K followers: \href{https://github.com/yegor256}{@yegor256})
and a manager of other programmers. At the moment, he is a director
of an R\&D laboratory in Huawei. His recent conference talks are in
\href{https://www.youtube.com/channel/UCr9qCdqXLm2SU0BIs6d_68Q}{his YouTube channel}.
He also published a \href{https://www.yegor256.com/books.html}{few books}
and wrote a \href{https://www.yegor256.com/contents.html}{blog} about software engineering
and object-oriented programming.
He previously taught a few courses in
\href{https://innopolis.university/}{Innopolis University} (Kazan, Russia)
and
\href{https://hse.ru}{HSE University} (Moscow, Russia),
for example,
\href{https://github.com/yegor256/ssd16}{SSD16 (2021)},
\href{https://github.com/yegor256/eqsp}{EQSP (2022)},
\href{https://github.com/yegor256/ppa}{PPA (2023)},
\href{https://github.com/yegor256/painofoop}{COOP (2023)},
\href{https://github.com/yegor256/pmba}{PMBA (2023)},
and
\href{https://github.com/yegor256/sqm}{SQM (2023)}
(all videos are available).

\textbf{Why this course?}\\
Writing code is the so-called ``hard skill'' that most students are very advanced in when they graduate from their bachelor's or master's programs. However, they usually lack experience in open source development, where teams are distributed, team members are not closely related to each other, and quality expectations are higher than in co-located teams working with proprietary codebases. Introducing students to the so-called ``soft skills'' may be beneficial to the projects they will join after graduation. Students will understand the dynamics and mechanics of software engineering much better and will contribute more fluently and effectively.

\textbf{What's the methodology?}\\
There are eight lectures, each a summary of best practices as seen by \href{https://www.kaicode.org}{Kaicode}, an open source festival organized and sponsored by Yegor since 2015. In laboratory classes, students either write research papers or submit pull requests to GitHub repositories suggested by the teacher. Groups that write the best papers will be encouraged to submit them to \href{http://www.icse-conferences.org/}{ICSE}, \href{https://www.esec-fse.org/}{ESEC/FSE}, or a similar A* conference (student or NIER tracks); the teacher will help them prepare the papers accordingly.

\newpage
\section*{Course Structure}

Prerequisites to the course (it is expected that a student knows this):

\begin{itemize}
\item How to write code
\item How to design software
\item How to use Git
\end{itemize}

After the course, students \emph{hopefully} will understand:

\begin{itemize}
\item How to make their bug reports appreciated?
\item How to make their pull requests merged?
\item How to reject a pull request politely?
\item How to become an active contributor of a large repository?
\item How to keep up with GitHub etiquette?
\item How to invite and motivate contributors?
\item How to deal with frustration during code reviews?
\item How to avoid stale pull requests (never merged)?
\item How to use GitHub Actions effectively?
\item How to format the \texttt{README.md} file?
\item How to control quality and avoid chaos in a public repository?
\item How to use GitHub account in lieu of a C.V.?
\item How to get 100 stars?
\item How to release in one click?
\item How to employ ChatGPT as a coding companion?
\item How to get 10K stars?
\item How to earn money via open source?
\end{itemize}

\newpage
\section*{Lectures \& Labs}

The following 80-minute lectures constitute the course:

\newlist{lectures}{enumerate}{10}
\setlist[lectures]{label*=\arabic*.}
\begin{lectures}
\item Debating
\item Reporting Bugs
\item Making Changes
\item Reviewing Changes
\item Setting Guidelines
\item Integrating
\item Releasing
\item Gaining Popularity
\end{lectures}

At laboratory classes, organized by a Teaching Assistant (TA),
students either make pull requests to
GitHub repositories suggested by the teacher, or write sections
for their research papers.

Most probably, one of the following repositories will be suggested
by the teacher for contribution during the course:
\href{https://github.com/yegor256/cam}{yegor256/cam} (Bash, Python),
\href{https://github.com/yegor256/rultor}{yegor256/rultor} (Java, XML),
\href{https://github.com/yegor256/qulice}{yegor256/qulice} (Java),
\href{https://github.com/cqfn/jpeek}{cqfn/jpeek} (Java, XML),
and
\href{https://github.com/objectionary/eo}{objectionary/eo} (Java, XSLT).

\newpage
\section*{Grading}

At the first lecture, students form groups of \textbf{2--3} people in each one (no exceptions!).
Each group picks a research topic from the list suggested by the teacher.

Each group writes a research paper in \LaTeX, according
to the \href{https://www.yegor256.com/2022/08/24/research-paper-template.html}{guideline}.
The length of the pager may not exceed four pages in
\href{https://ctan.org/pkg/acmart}{acmart/sigplan} 10pt format
(including references and appendices).
The paper must be presented to the teacher \angry{incrementally}, section by section:
\begin{inparaenum}[1)]
\item Method,
\item Related Work,
\item Results,
\item Discussion,
\item Conclusion,
\item Introduction,
and
\item Abstract.
\end{inparaenum}
Once a section is \angry{approved} by the teacher, the next section may be presented for review.

After a presentation of a section, the teacher may ask the group to \angry{stop}
working with the paper. In this case, no sections may be presented for review any more: they all will be rejected.
This decision is \angry{subjectively} made by the teacher and will \angry{not} be explained
to the students, however the following may contribute to such a
negative decision:
\begin{inparaenum}[a)]
    \item ChatGPT,
    \item plagiarism,
    \item negligence,
    and
    \item laziness.
\end{inparaenum}

When the Abstract is accepted by the teacher, a group may ask a student from another
group to review their paper, according to
\href{https://www.yegor256.com/2023/12/17/how-to-review-research-paper.html}{this guideline}.
The review must be accepted by the TA.

There is no exam at the end of the course. Instead,
each student earns points for the following results:\\
\renewcommand{\arraystretch}{1}
\begin{tabular}{lrr}
Result & Points & Limit \\
\hline
Attended a lecture & +2 & 8 \\
Attended a lab & +2 & 8 \\
Merged a pull request to a suggested repo & +4 & 48 \\
Reviewed a paper of another group & +3 & 9 \\
``Method'' section (\href{https://www.yegor256.com/2023/10/11/method-of-research.html}{guideline}) & +8 \\
``Related Work'' section (\href{https://www.yegor256.com/2023/09/29/how-to-write-related-work-section.html}{guideline}) & +12 \\
``Results'' and ``Discussion'' sections (\href{https://www.yegor256.com/2023/12/11/results-and-discussion.html}{guideline}) & +12 \\
``Conclusion'' & +6 \\
``Introduction'' section & +6 \\
``Abstract'' and Title & +4 \\
\end{tabular}

Then, 55+ points mean ``A,'' 47+ mean ``B,'' and 23+ mean ``C.''

An online lecture is counted as ``attended'' only if a student was personally
presented in Zoom for more than 75\% of the lecture's time. Watching the
lecture from the computer of a friend doesn't count.

% \newpage
% \section*{Learning Material}

% The following books are highly recommended to read (in no particular order):

% \begin{multicols}{2}\small\raggedright
% \nospell{Rita Mulcahy}, \emph{PMP Exam Prep, 9th Edition}, 2018\\[3pt]
% \nospell{Rita Mulcahy}, \emph{Risk Management}, 2010\\[3pt]
% \nospell{Karl Wiegers} et al., \emph{Software Requirements}\\[3pt]
% \nospell{Alistair Cockburn}, \emph{Writing Effective Use Cases}\\[3pt]
% \nospell{Steve McConnell}, \emph{Software Estimation: Demystifying the Black Art}\\[3pt]
% \nospell{Frederick Brooks Jr.}, \emph{Mythical Man-Month, The: Essays on Software Engineering}\\[3pt]
% \nospell{David Thomas et al.}, \emph{The Pragmatic Programmer: Your Journey To Mastery}\\[3pt]
% \nospell{Robert C. Martin}, \emph{Clean Code: A Handbook of Agile Software Craftsmanship}\\[3pt]
% \nospell{Jez Humble} et al., \emph{Continuous Delivery: Reliable Software Releases through Build, Test, and Deployment Automation}\\[3pt]
% \nospell{Michael T. Nygard}, \emph{Release It!: Design and Deploy Production-Ready Software}\\[3pt]
% \nospell{Yegor Bugayenko}, \emph{Code Ahead}, 2019\\[3pt]
% Blog posts of \nospell{Yegor Bugayenko}, \href{https://www.yegor256.com/tag/management}{on his blog}\\[3pt]
% \end{multicols}

\end{document}
