% SPDX-FileCopyrightText: Copyright (c) 2023-2026 Yegor Bugayenko
% SPDX-License-Identifier: MIT

\documentclass{article}
\usepackage{../osbp}
\newcommand*\thetitle{Releasing}
\begin{document}

\lnTitlePage{7}{8}{VTsbvZSMwYE}

\lnQuote
  [\nospell{Andre Van Der Hoek}]
  {andre-van-der-hoek}
  {Simply \ul{making available} and retrieving interdependent components individually \ul{neither} facilitates independent software development \ul{nor} encourages widespread use of large systems of systems.}
  {van1997software}

\lnQuote
  [\nospell{G{\"u}nther Ruhe}]
  {gunther-ruhe}
  {Release planning is typically done \textit{ad hoc} and \ul{not} based on sound \ul{models and methodologies}. This is even the case when planning involves several hundred features.}
  {ruhe2005art}

\lnThought{Release when it's different, not when it's good.}

\lnQuote
  [\nospell{Kazu Okumoto}]
  {kazu-okumoto}
  {\textbf{\nospell{Goel-Okumoto} model}: An important problem of practical concern is the determination of the point when testing should \ul{stop} and the system can be considered ready for release, that is, the determination of the \ul{software release time}. Two criteria are investigated: software reliability and total expected cost.}
  {okumoto1979optimum}

\lnQuote
  [\nospell{Shigeru Yamada}]
  {shigeru-yamada}
  {This paper extends the problem of \citet{okumoto1979optimum} by evaluating \ul{both criteria} simultaneously. We discuss optimal software release policies which minimize a total average software cost under the constraint of satisfying a software reliability requirement.}
  {yamada1985cost}
\lnPitch{
  \begin{multicols}{2}
  \includegraphics[width=.65\linewidth]{time-vs-cost.png}
  \par\columnbreak\par
  ``The optimum software release time is the testing time which comes closest to satisfying some pre-specified software reliability.''
  \lnSource{yamada1985cost}
  \end{multicols}}

\lnQuote
  [\nospell{Steve McConnell}]
  {steve-mcconnell}
  {The question of whether to release software is a treacherous one. The answer must teeter on the line between releasing poor quality software early and releasing high quality software late. The questions of `Is the software good enough to release now?' and `When will the software be good enough to release?' can become critical to a company’s survival.}
  {mcconnell1998}

\lnQuote
  [\nospell{Nasif Imtiaz}]
  {nasif-imtiaz}
  {We find that the open source packages are typically fast in releasing security fixes, as the median release comes within \ul{4 days} of the corresponding security fix. However, 25\% of the releases still have a delay of at least \ul{20 days}.}
  {imtiaz2022open}

\lnThought{Fully automate the release process, avoiding any manual intervention.}

\lnQuote
  [\nospell{Michael Nygard}]
  {michael-nygard}
  {Releases should be about as big an event as getting a \ul{haircut}. There's an added benefit of frequent releases: it forces you to get really good at doing releases and deployments.}
  {nygard2007release}

\lnQuote
  [\nospell{Jez Humble}]
  {jez-humble}
  {Over time, deployments should tend towards being fully automated. There should be \ul{two tasks} for a human being to perform to deploy software into a development, test, or production environment: to pick the version and environment and to press the `deploy' button.}
  {humble2010continuous}

\lnThought{Release frequently.}

\lnQuote
  [\nospell{Victor Basili}]
  {victor-basili}
  {If improving productivity is the main concern, then it may be wise to try to \ul{avoid} scheduling \ul{small} error correction releases. Instead the manager should try, when possible, to \ul{package} small error corrections in a release with larger enhancements.
  \pptPin{\raggedleft\colorbox{red}{\color{white}Respectfully} \\ \colorbox{red}{\color{white}disagree!}\par}}
  {basili1996understanding}

\lnQuote
  [\nospell{Foutse Khomh}]
  {foutse-khomh}
  {We found that (1)~with shorter release cycles, users do not experience significantly more post-release bugs and (2)~bugs are fixed faster, yet (3)~users experience these bugs earlier during software execution (the program crashes earlier).}
  {khomh2012faster}

\lnThought{Use SemVer.}

\lnPitch{
  \begin{multicols}{2}
  \includegraphics[width=.9\linewidth]{semver.png}
  \par\columnbreak\par
  ``The implication of semantic versioning is that clients may rely on dependencies subject to flexible version constraints, like \texttt{1.2.*}. Such a client may safely upgrade to new micro versions (e.g., from \texttt{1.2.3} to \texttt{1.2.4}), fully automated.''
  \lnSource{lam2020putting}
  \end{multicols}}

\lnThought{Generate release notes automatically.}

\lnPitch{
  \begin{multicols}{2}
  \includegraphics[width=.9\linewidth]{notes.png}
  \par\columnbreak\par
  ``Automatically generated release notes provide an automated alternative to manually writing release notes for your GitHub releases. With automatically generated release notes, you can quickly generate an overview of the contents of a release.''
  {\par\scriptsize\url{https://docs.github.com/en/repositories/releasing-projects-on-github/automatically-generated-release-notes}\par}
  \end{multicols}}

\lnPitch{
  \pptBanner{Some Release Notes Generators:}
  \begin{itemize}\setlength\itemsep{0em}
    \item \href{https://clickup.com/features/ai/release-notes-generator}{ClickUp}
    \item \href{https://www.taskade.com/generate/programming/release-notes}{Taskade}
    \item \href{https://zeda.io/feature/release-note-ai}{Zeda}
    \item \href{https://www.aha.io/blog/introducing-ai-powered-release-notes}{Aha}
    \item \href{https://www.releasesnotes.dev/}{ReleasesNotes}
    \item \href{https://scribehow.com/tools/product-release-note-generator}{ScribeHow}
    \item \href{https://www.released.so/}{Released}
    \item \href{https://github.com/marketplace/ai-github-release-notes}{ai-github-release-notes}
  \end{itemize}
  {\scriptsize Try Google search with ``generate release notes with AI''\par}}

\lnQuote
  [\nospell{Jianyu Wu}]
  {jianyu-wu}
  {We find that: 1) RN producers are more likely to miss information than to include incorrect information, especially for breaking changes; 2) improper layout may bury important information and confuse users; 3) many users find RNs inaccessible due to link deterioration, lack of notification, and obfuscate RN locations; 4) automating and regulating RN production remains challenging despite the great needs of RN producers.}
  {wu2022demystifying}

\lnThought{Publish binaries.}

\lnPitch{
  \pptBanner{Some Artifact Publishing Platforms:}
  \begin{itemize}\setlength\itemsep{0em}
    \item \href{https://central.sonatype.com/}{Maven Central} for Java, Kotlin, Scala, Groovy, etc.
    \item \href{https://www.npmjs.com/}{Npm} for JavaScript
    \item \href{https://pypi.org/}{PyPi} for Python
    \item \href{https://rubygems.org/}{RubyGems} for Ruby
    \item \href{https://crates.io/}{Crates} for Rus
  \end{itemize}}

% \lnThought{Be aware of malware and protestware.}

% \lnQuote
%   [\nospell{Marc Cheong}]
%   {marc-cheong}
%   {By providing \ul{guidance} on ethical decision making and highlighting alternative methods for expressing concerns, ethics education can play a vital role in promoting a more balanced and responsible approach to protestware within the FOSS community.}
%   {cheong2023ethical}

\end{document}
