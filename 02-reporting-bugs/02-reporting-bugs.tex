% (The MIT License)
%
% Copyright (c) 2023-2024 Yegor Bugayenko
%
% Permission is hereby granted, free of charge, to any person obtaining a copy
% of this software and associated documentation files (the 'Software'), to deal
% in the Software without restriction, including without limitation the rights
% to use, copy, modify, merge, publish, distribute, sublicense, and/or sell
% copies of the Software, and to permit persons to whom the Software is
% furnished to do so, subject to the following conditions:
%
% The above copyright notice and this permission notice shall be included in all
% copies or substantial portions of the Software.
%
% THE SOFTWARE IS PROVIDED 'AS IS', WITHOUT WARRANTY OF ANY KIND, EXPRESS OR
% IMPLIED, INCLUDING BUT NOT LIMITED TO THE WARRANTIES OF MERCHANTABILITY,
% FITNESS FOR A PARTICULAR PURPOSE AND NONINFRINGEMENT. IN NO EVENT SHALL THE
% AUTHORS OR COPYRIGHT HOLDERS BE LIABLE FOR ANY CLAIM, DAMAGES OR OTHER
% LIABILITY, WHETHER IN AN ACTION OF CONTRACT, TORT OR OTHERWISE, ARISING FROM,
% OUT OF OR IN CONNECTION WITH THE SOFTWARE OR THE USE OR OTHER DEALINGS IN THE
% SOFTWARE.

\documentclass{article}
\usepackage{../osbp}
\newcommand*\thetitle{Reporting Bugs}
\begin{document}

\plush{\osbpTitlePage{2}{nLcnE4N3Nz4}}

\qte
  [\nospell{Joel Spolsky}]
  {joel-spolsky}
  {Every good bug report needs exactly three things: \ul{steps} to reproduce, what you \ul{expected} to see, and what you \ul{saw} instead.}
  {spolsky2000bugs}

\qte
  [\nospell{Nicolas Bettenburg}]
  {nicolas-bettenburg}
  {Well-written bug reports are likely to get \ul{more attention} among developers than poorly written ones... Steps to reproduce and stack traces are most useful in bug reports. The most severe problems encountered by developers are errors in steps to reproduce, incomplete information, and wrong observed behavior.}
  {bettenburg2008makes}

\qte
  [\nospell{Tommaso Dal Sasso}]
  {tommaso-dal-sasso}
  {The elements considered to be harder to provide [while reporting bugs] are the \ul{entity} (e.g., class, file) that likely contains the defect, the \ul{steps} to reproduce the failure, and a \ul{test case} showing the defect.}
  {dal2016makes}

\thought{Expect any program to have an unlimited number of bugs}

\qte
  {../02-reporting-bugs/hetzel-book}
  {We can \ul{never} be certain that a testing system is correct. These theoretical limit tells us that there will \ul{never} be a way to be sure we have a perfect understanding of what a program is supposed to do (the expected or required results) and that any testing system we might construct will always have some possibility of failing. In short, we \ul{cannot} achieve 100 percent confidence no matter how much time and energy we put into it!}
  {hetzel1993complete}

\qte
  [\nospell{Myers, Glenford J.}]
  {glenford-myers}
  {You cannot test a program to guarantee that it is error free... It is impractical, often impossible, to find \ul{all} the errors in a program.}
  {myers2012art}

\thought{Don't ask questions or suggest features, report bugs instead}

\pitch{\pptBanner{All of These Are Bugs:}
\begin{itemize}\setlength\itemsep{0em}
\item Lack of functionality
\item Lack of tests
\item Lack of documentation
\item Suboptimal implementation
\item Design inconsistency
\item Naming is weird
\item Unstable test
\end{itemize}
More about it by \citet{bugayenko2018blog0206}.}

\qte
  [\nospell{Reis Christian}]
  {reis-christian}
  {Though it commonly has a prejorative connotation, in the Mozilla Project the term \textbf{bug} is used to refer to any field request for modification in the software, be it an \ul{actual} defect, an enhancement, or a change in functionality. (``bug-driven development'')}
  {reis2002overview}

\qte
  [\nospell{Kim Herzig}]
  {kim-herzig}
  {In a manual examination of more than 7,000 issue reports from five open-source projects, we found 33.8\% of all bug reports to be \ul{misclassified} threatening bug prediction models, confusing \ul{bugs} and \ul{features}: On average, 39\% of files marked as defective actually never had a bug.}
  {herzig2013s}

\thought[bugayenko2018blog0724]{Reward yourself for each reported bug}

\thought{Report strictly one problem per ticket}

\qte
  [\nospell{John Anvik}]
  {john-anvik}
  {People play different roles as they interact with reports in a bug repository. The person who submits the report is the \ul{reporter} or the \ul{submitter} of the report. The \ul{triager} is the person who decides if the report is meaningful and who assigns responsibility of the report to a developer. The one that resolves the report is the \ul{resolver}. A person that contributes a fix for a bug is called a \ul{contributor}.}
  {anvik2006should}

\thought[bugayenko2018blog0424]{Exaggerate for effect}

\pitch{\pptBanner{Do This, While Reporting Bugs:}
\begin{itemize}\setlength\itemsep{0em}
\item Stay cool
\item Exaggerate
\item Victimize yourself
\item Push them
\item Show efforts
\item Look engaged
\item Look altruistic
\item Aggregate (not!)
\end{itemize}
More about it by \citet{bugayenko2018blog0424}.}

\thought[bugayenko2022blog0329]{Simplify until it's impossible to simplify any further}

\pptBanner{Bugs Occam's Razor}
\begin{pptWide}{2}
{\small\begin{ffcode}
Here is my code:

a := 7
a := a + 5 - 3
a := a / 3
print a

It doesn't work as expected.
It prints 4, but it should
print 3.
\end{ffcode}
}
\par\columnbreak\par
{\small\begin{ffcode}
Here is my code:

a := 7
a := a - 3
print a

It doesn't work as expected.
It prints 7, but it should
print 4.
\end{ffcode}
}
Most probably, the subtracting operator doesn't do anything --- this is the bug.
\end{pptWide}
\plush{}

\thought[bugayenko2014blog1124]{Be a prosecutor, not an advocate}

\thought[bugayenko2014blog1124]{Don't give up until some changes are made to the code base}

\thought[bugayenko2023blog0725]{Prefer a disabled test in lieu of a bug report}

\pptBanner{Why not?}
\begin{pptWide}{2}
{\small\begin{ffcode}
// @todo #42 This test is disabled
//  because the fibo() doesn't work
//  correctly with this input, returning
//  17711 instead of 28657. Fix it.
#[test]
(*@\textcolor{orange}{\#[ignore]}@*)
fn calculates_23rd_fibonacci_number() {
  let x = fibo(23);
  assert_eq!(28657, x);
}
\end{ffcode}
}
\par\columnbreak\par
``Such a PR serves as both a bug report (this is what the text of the puzzle will be turned into, once the PR is merged) and a test that reproduces the problem. It will be more than \ul{welcome} by the repository maintenance team. This kind of PR saves the time they would spend creating a unit test. Also, it saves your time for creating a bug report, as it will be created automatically by the \ul{puzzles discovery tool}.''~\citep{bugayenko2023blog0725}
\end{pptWide}
\plush{}

\pitch{\pptBanner{Repositories With a Lot of Bugs (9 Feb 2024)}
{\ttfamily\small\begin{tabular}{lr}
\toprule
Github Repository & Issues \\
\midrule
\href{https://bugzilla.mozilla.org/}{mozilla bugzilla} & 1,8M+ \\
\href{https://gitlab.com/gitlab-org/gitlab/-/issues}{gitlab-org/gitlab} & 172,462 \\
\href{https://github.com/flutter/flutter}{flutter/flutter} & 79,386 \\
\href{https://github.com/kubernetes/kubernetes}{kubernetes/kubernetes} & 42,627 \\
\href{https://github.com/tensorflow/tensorflow}{tensorflow/tensorflow} & 36,776 \\
\href{https://github.com/moby/moby}{moby/moby} & 19,367 \\
\bottomrule
\end{tabular}}\par
All repositories are open source.}


\end{document}
